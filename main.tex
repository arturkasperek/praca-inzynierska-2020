\documentclass[twoside]{projektInzynierskiMS1}
\usepackage{polski}
\usepackage[utf8]{inputenc}
\usepackage{amsmath}
\usepackage{graphicx}

\graphicspath{ {./images/} }
%\drukJednostronny

%% tytuł promotor iautor (\title to komenda standardowa)
\title{Automatyzacja procesów w przemyśle IT}
\promotor{dr inż. Zdzisław Sroczyński}


%% każdy autor musi mieć 3 argumenty: imię nazwisko, nr albumu, opis wkładu
\autor{Artur Kasperek}{1234}
\autor{Michał Płonka}{1234}
\autor{Patryk Musiol}{1234}

%% dedykacja mile widziana
% \dedykacja{To jest\\dedykacja}
%\NumeryNaPoczatku
%% numeracja wzorów tu włączona typu (1.2.3), ta druga to typu (1.2), domyślnie typu (1)
%\subsectionWzory
% \sectionWzory  

%\rozdzialy


%\literowaNumeracjaDodatkow %% włączy numerację dodatków literami
%\rzymskaNumeracjaDodatkow  %%włączy numerację dodatków liczbami rzymskimi

%% wyłączenie wyjaśnień:
\bezWyjasnien

%% standardowe komendy \newtheorem  działają jak woryginale
\newtheorem{tw}{Twierdzenie}%[subsection]
\newtheorem{twa}{Twierdzenie}%[section]
\newtheorem{dd}{Definicja}%[subsection]

\begin{document}
TODO dodać wstęp


\section{Automatyzacja a branża IT}
\subsection{Charakteryzacja branży IT}
W 2001 roku doszło do jednej z bardziej znaczących publikacji dla szerokopojętego biznesu IT. W roku tym został opublikowany \textit{Manifesto for Agile Software Development} autorstwa między innymi Kenta Becka, Roberta C. Martina oraz Martina Fowlera. Manifest ten opisywał rewolucyjne jak na tamte czasy praktyki \cite{AgileManifesto}:
\begin{itemize}
    \item Satysfakcja klienta dzięki wczesnemu i ciągłemu dostarczaniu oprogramowania.
    \item Zmiany wymagań mile widziane nawet na późnym etapie programowania.
    \item Dostarczanie działającej wersji oprogramowanie często (bardziej tygodnie niż miesiące).
    \item Bliska kooperacja między programistami a ludziami zajmującymi się biznesem
    \item Projekty powstają wokół zmotywowanych osób, którym należy ufać.
    \item Komunikacja w cztery oczy jest najlepszą formą komunikacji
    \item Działający produkt jest najlepszym wskaźnikiem postępu prac.
    \item Zrównoważony rozwój, pozwalający na utrzymania stałego tempa.
    \item Ciągła dbałość o doskonałość techniczną i dobry design.
    \item Prostota - sztuka projektowania systemu bez dużej komplikacji systemu.
    \item Najlepsze architektury, wymagania i designy powstają dzięki samoorganizującym się zespołom
    \item Zespół regularnie zastanawia się, jak zwiększyć skuteczność, i odpowiednio się dostosowuje. 
\end{itemize}
Propozycje przedstawione przez autorów manifestu były dużą zmianą w stosunku do podejścia które było używane powszechnie w tamtych czasach.
\par
W latach 80 oraz 90 powszechnie stosowaną techniką była metodologia Waterfall zobrazowana na Rysunku \ref{fig:waterfall}.
\begin{figure}[htbp]
    \centering
    \includegraphics[width=10cm]{waterfall.png}
    \caption{Metodologia Waterfall}
    \label{fig:waterfall}
\end{figure}
Poszczególne etapy projektowe były wykonywane tylko raz podczas procesu tworzenia oprogramowania. Z tego faktu wszelakie zmiany na późniejszym etapie projektowym były trudne w realizacji. Konkurencja na rynku oprogramowania komputerowego była na tyle niewielka, że producenci oprogramowania nie musieli przejmować się zanadto uwagami od użytkowników - to sprawiało, że Waterfall spełniał swoje zadania. 
\par
Pierwsze dziesięciolecie XXI wieku spopularyzowało jedno z największych osiągnięć ludzkości - Internet. Nowa rzeczywistość w której ludzkość coraz więcej czasu spędza przed urządzeniami elektronicznymi postawiła przez twórcami oprogramowania nowe wymagania. Dodatkowo coraz większe grono przedsiębiorców zaczyna dostrzegać w produkcji oprogramowania zyski. Użytkownicy zaczynają coraz bardziej spoglądać na przyjemny dla oka wygląd oprogramowania oraz jego prostotę. Metodyka zaproponowana przez autorów manifestu Agile zdaje się świetnie wpisywać w wizję tworzenia oprogramowania na miarę XXI wieku.
\par
Jedną z bardziej znanych implementacji Agile jest metodologia o nazwie Scrum.
\begin{figure}[htbp]
    \centering
    \includegraphics[width=10cm]{scrum.png}
    \caption{Framework Scrum}
    \label{fig:scrum}
\end{figure}
Zakłada się w niej, że oprogramowanie powstaje w procesie kolejnych inkrementacji. Każda iteracja jest nazywana sprintem. Sprint ma z góry zdefiniowane ramy czasowe w których będzie on trwał. Na podstawie różnych czynników biznesowych opiekun projektu decyduje które zadania powinny trafić do danego sprintu, a które są mniej priorytetowe i mogą pozostać w tzw. backlogu. Efektem końcowym danego sprintu jest działający produkt, który jest wzbogacony o rzeczy dodane podczas trwania sprintu. Scrum sam w sobie nie narzuca ile powinien trwać dany sprint, czy też jaki system powinien być stosowany do śledzenia zadań. Wszystko zależy od preferencji danego zespołu programistycznego. Integralną częścią każdego sprintu jest retrospektywa. Na tym spotkaniu zespół dyskutuje jakie zmiany należy dokonać w procesie by uefektywnić pracę. Dzięki elastycznemu podejściu i możliwości ulepszania procesu Scrum wydaje się być dobrym rozwiązaniem dla zespołów które wypuszczają oprogramowanie regularnie oraz zmieniają je na podstawie opinii użytkowników.
\subsection{Agile a automatyzacja}
Spełnienie wymagań wymienionych w manifeście Agile wydaje się być trudne w kontekście częstego wypuszczania działającej wersji. Oczekuje się tego by regularnie zespół deweloperski publikował działającą wersję podglądową oprogramowania dla osób nietechnicznych. Problem ten można rozwiązać na conajmniej dwa sposoby:
\begin{itemize}
    \item Manualny - Członek zespołu deweloperskiego regularnie według wymagań zajmuje się budowaniem wersji podglądowej oraz udostępnia ją osobom zainteresowanym
    \item Automatyczny - Zespół deweloperski ustawia automatyczne procesy które na serwerze budującym tworzą wersję podglądową aplikacji oraz publikują ją dla osób zainteresowanych
\end{itemize}
Proces automatyczny jest preferowanym sposobem publikacji oprogramowania. Ma on kilka zalet nad sposobem manualnym. Nie tracimy czasu specjalisty który musiałby poświęcić go na zbudowanie i publikację aplikacji. Drugą zaletą jest fakt, że serwer za każdym razem robi te same kroki podczas procesu budowania. Tym sposobem wykluczamy możliwość popełnienia błędu przez człowieka.
\par
Dobre praktyki związane z częstym budowaniem podglądowej wersji oprogramowania są określane jako DevOps. Len Bass, Ingo Weber oraz Liming Zhu w swojej książce \cite{DevOpsBook} określają DevOps jako zbiór praktyk których celem jest zmniejszenie czasu publikacji zmian na serwerze produkcyjnym przy jednoczesnej trosce o wysoką jakość. W praktyce często członkiem zespołu deweloperskiego jest tzw. DevOps. Jego zadaniem jest automatyzacja wszelakich procesów oraz często także utrzymanie środowiska produkcyjnego. Osoba na tym stanowisku powinna się cechować dobrą znajomością systemu operacyjnego który jest używany na serwerach produkcyjnych oraz deweloperskich. Ponadto powinna być zorientowana w różnych rozwiązaniach chmurowych które współcześnie są coraz częściej używane.
\subsection{Git - kamień milowy dla deweloperów}
Ciężko byłoby sobie wyobrazić obraz dzisiejszego przemysłu IT gdyby nie system kontroli wersji Git. Jego autorem jest Linus Torvalds. Torvalds stworzył go jako dodatkowy projekt który miał pomóc w pisaniu jądra Linuxa. Dzięki Git'owi każdy członek zespołu deweloperskiego ma dostęp do wspólnego repozytorium gdzie każdy może publikować swoje zmiany. Jedną z ważniejszych funkcji Git'a jest możliwość tworzenia własnych rozgałęzień kodu gdzie dany programista wysyła swoje zmiany. Dalej w procesie merge'owania jest możliwe połączenie zmian danego dewelopera z kodem innych programistów. Dzięki tej cesze Git nadaje się świetnie do wszelakich projektów programistycznych w których pracuje kilku programistów równolegle. Z biegiem lat Git stał się standardem.
\par
Opisać git workflow TODO!!
\par
W obecnych czasach popularne są rozwiązania SaaS takie jak GitHub, GitLab czy też BitBucket. Dzięki takiemu rozwiązaniu nie musimy się przejmować utrzymaniem własnego serwera Git, dodatkowo platformy SaaS zapewniają nam wszelakie aktualizacje które usprawniają system. Platformy takie jak GitHub zapewniają narzędzie które ułatwiają proces tworzenia oprogramowania. Narzędzie te są dopasowane by działać dobrze z naszym repozytorium. Przykładem takiego rozwiązania jest \textit{GitHub Pages}, jest to technologia która pozwala publikować stronę www na podstawie plików które są częścią repozytorium. Użytkownik definiuje na której gałęzi oraz w którym folderze znajdują się pliki z stroną internetową. Od tego momentu GitHub automatycznie stwarza nam stronę internetową dostępną pod subdomeną \textit{nazwauzytkownika.github.io}.
\subsection{Ciągła integracja}
\subsection{Ciągłe dostarczanie}
\subsection{Ciągłe dowożenie}
\subsection{Serwery automatyzujące PaaS kontra self-hosted}
\subsection{GitOps - czym jest?}
\section{Wirtualizacja i orkiestracja}
myślę, że opisanie tego jak wirtualizacja pomaga w CI/CD zasługuje na swój rozdział. Szczególnie mógłbym ten rozdział poświęcić dockerowi z faktu, że zdominował rynek. Warto tutaj byłoby napisać też o rozwiązaniach które były używane w przeszłości a zostały wyparte jak vagrant. Dodatkowo w rozdziale mozemy opisać na czym polega orkiestracja - Kubernates, docker swarm
\section{Hudson oraz Jenkins - klasyczne narzędzia do CI/CD}
\subsection{Inżynieria oprogramowania}
Kolejny rozdział tej pracy inżynierskiej poświęcimy na opisanie aspektów automatyzacji w procesach tworzenia oprogramowania. 
Jako wprowadzenie do tego rozdziału chcielibyśmy na podstawie książki "The Phoenix Project: A Novel about IT, DevOps, and Helping Your Business Win" autorstwa Gene Kim i Kevin Behr\cite{PhoenixProject} pokazać jak popularyzacja komputerów spowodowała zapotrzebowanie na nowe technologie. 
Z biegiem czasów gdy komputery stawały się coraz to bardziej popularne, powstawało coraz więcej aplikacji czy to desktopowych czy to webowych. Coraz więcej programistów i aplikacji pojawiało się na rynku. Powstawało wiele firm tworzących oprogramowanie i w związku z dużą konkurencyjnością, firmy wymyślały nowe sposoby poprawy wdrażania oprogramowania, sprawdzania jakości kodu, dostępności infrastruktury oraz poprawy wydajności kodu, po to by wyróżnić się na konkurencyjnym rynku. Coraz to większą rolę w środowisku IT zaczęli odgrywać  DevOpsi, czyli jak podają autorzy książki, osoby tworzące równowagę pomiędzy działami wytwarzania oprogramowania i zarządzania systemami.  


\subsection{DevOps} 
Do głównych zadań inżynierów DevOps należy:
\begin{itemize}
    \item projektowanie strategi kontroli wersji,
    \item wdrożenie i integracja kontroli źródeł,
    \item implementacja i zarządzanie infrastrukturą build'owania,
    \item wdrażanie przepływu kodu,
    \item zarządzanie konfiguracją aplikacji i jej tajnymi danymi.
\end{itemize}

Autorzy książki opisują duży chaos w zarządzaniu kodem w czasach kiedy wszystkie te koncepty nie były jeszcze tak popularne jak dzisiaj. Opisują środowiska przechowywania kodu jako miejsca mało zadbane i prowadzone bez głębszego pomysłu.

DevOps to podejście do rozwoju oprogramowania, które obejmuje ciągły rozwój, ciągłe testowanie, ciągłą integrację, ciągłe wdrażanie i ciągłe monitorowanie oprogramowania w całym cyklu jego życia. Jest to proces przyjęty przez wszystkie najlepsze firmy w celu opracowania wysokiej jakości oprogramowania i skrócenia czasu tworzenia produktu, co przekłada się na większą satysfakcję klientów, czego każda firma poszukuje.
Inżynierowie DevOps codziennie korzystają z wielu narzędzi jak Kibana czy Splunk do monitorowania aplikacji, Git czy Mercurial do zarządzania kodem, Puppet, Ansible bądź Chef do zarządzania konfiguracją i wiele innych. 
W dalszej części pracy skupimy się na narzędziu, które subiektywnym zdaniem autorów tej pracy najlepiej obrazuje codzienne zadania automatyzujące inżynierów DevOps czyli Jenkins.
Na potwierdzenie słuszności naszego wyboru warto dodać, że narzędzie to w 2011 roku wygrało nagrodę Bossie (Best of Open Source Software Award) oraz w 2014 prestiżową nagrodę Geek Choice.  

\subsection{Projekt}

Celem projektu jest przybliżenie możliwości automatyzacji na podstawie nowoczesnego narzędzia codziennie wykorzystywanego w świecie IT. Tym narzędziem będzie Jenkins. 
Na potrzeby tego projektu stworzymy również prostą aplikację Spring Boot, która będzie implementowała podstawowe założenia API RESTful. Jenkins oraz aplikacja w nawiązaniu do poprzedniego rozdziału tej pracy będą uruchomione w kontenerach. Celem tego zabiegu jest zaprezentowanie działania tego narzędzia. 

\subsection{Jenkins}

Jest to projekt open-source napisany całkowicie w języku Java. Jenkins wykonuje szereg zadań by osiągnąć założenia ciągłej integracji poprzez automatyzację części związanych z budowaniem, testowaniem i wdrażaniem. To sprawia, że developerzy mogą ciągle pracować nad ulepszaniem produktu nad którym pracują. Ponadto jest to system, który działa na servletowych kontenerach, jak na przykład Apache Tomcat. 
Jenkins automatyzuje budowanie aplikacji, dzięki czemu developerzy są w stanie wcześnie wykrywać błędy w swoim kodzie. Do głównych zalet Jenkinsa zdecydowanie można zaliczyć społeczność, która się wokół Jenkinsa przez wiele lat działalności zbudowała. Jest to narzędzie nie tylko łatwo rozszerzalne, ale również posiada wiele zaimplementowanych wtyczek. 
Kilka przykładów zastosowania tego oprogramowania:
\begin{itemize}
    \item budowanie aplikacji przy pomocy narzędzi do buildowania jak Gradle, Maven czy inne, 
    \item automatyzacja testów  (Nose2, PyTest, Robot, Selenium i wiele innych),
    \item wykorzystywany do testowania skryptów (bash, bat, zsh, inne),
    \item raportowanie, czyli na przykład wyświetlanie wyników testów.
\end{itemize}

Na czas pisania tej pracy Jenkins posiada ponad 1500 wtyczek stworzonych przez społeczność, dzięki którym doświadczenie z korzystania z narzędzia oraz aktywności związane z budowaniem, wdrażaniem i automatyzacją projektu stają się lepsze. 

\subsection{Historia}

Jenkins nie zawsze nosił nazwę taką jak dziś. Został stworzony przez pracownika Sun Microsystems Kohsuke Kawaguchi w lato 2004 roku, a pierwsze wydanie nastąpiło w styczniu 2005 roku pod nazwą "Hudson". Oprogramowanie występuje pod nazwą Jenkins od 2011 roku po tym jak firma Sun Microsystems została wykupiona przez firmę Oracle. Na początku Hudson i Jenkins były tworzone osobno, ale po przejęciu firmy zarząd postanowił połączyć oba projekty i zachować nazwę Jenkins, gdyż posiadał on znacząco większą społeczność niż projekt Hudson. Dzisiaj wsparcie dla projektu Hudson nie jest oficjalnie prowadzone. 

\subsection{Architectura}

Aby dobrze zrozumieć jak działa narzędzie, w tym rozdziale opiszemy co się stanie jeśli developer zapisze zmiany na repozytorium, przedstawimy przykładową implementację metodologii ciągłej integracji/ciągłego wdrażania w Jenkinsie oraz opiszemy jak wygląda architektura Master-Slave.

Według Johna Smart, czyli autora książki "Jenkins: The Definitive Guide"\cite{Jenkins} istnieje kilka kroków, które opisują jak działa komunikacja między elementami w Jenkinsie: 
\begin{itemize}
    \item inżynier zmienia kod źródłowy aplikacji i zapisuje zmiany do repozytorium,
    \item repozytorium jest regularnie sprawdzane przez serwer Jenkins CI i w razie jakiś zmian ten sam serwer pobiera je do dalszej pracy,
    \item w następnym kroku jest sprawdzane czy zapisane zmiany "przechodzą". Build  jest wykonywany i jeśli nie było żadnych błędów generowany jest plik wykonywalny. Jeśli pojawią się jakieś błędy, tworzony jest email z linkiem do logów builda,
    \item w przypadku gdy build był udany, plik wykonywalny jest wdrażany na środowisku testowym. Ten krok pomaga zrealizować krok ciągłego testowania ponieważ plik wykonywalny przechodzi przez wiele testów automatycznych. Jeśli są problemy w którymś z testów, programiści również są o tym informowani,
    \item jeśli nie ma problemów podczas buildu, integracji czy testowania - zmiany są automatycznie wdrażane na środowisko produkcyjne.
\end{itemize}



Często się zdarza, że pojedynczy serwer może nie wystarczyć. Na przykład:
\begin{itemize}
    \item testy muszą być wykonane na różnych środowiskach,
    \item pojedynczy serwer nie jest wstanie obsłużyć ruchu, który jest wymagany w wielkich systemach.
\end{itemize}

W tych przypadkach wykorzystywana jest architektura Master-slave, wspomniana krótko na początku tego rozdziału. Zostanie ona przedstawiona dokładniej w dalszej części tej pracy. 

Architektura master-slave jest używana do zarządzania rozszerzonymi buildami. Komunikacja między serwerem mastera i slave'a odbywa się poprzez protokół TCP/IP. 

\subsection{Master}

To jest główny serwer Jenkinsa. Do jego głównych zadań należą:
\begin{itemize}
    \item zorganizowanie "jobów" builda,
    \item wybór odpowiedniego slave'a,
    \item monitorowanie slave'ów i w razie potrzeby włączanie/wyłączanie ich,
    \item raportowanie wyników builda do developerów.
\end{itemize}

Master również może zostać wykorzystywany bezpośrednio do wykonywania jobów ale rekomendowane jest, żeby były one wykonywane na slaveach.

\subsection{Slave}

Slave'ami nazywamy zewnętrzną maszynę połączoną z Masterem. Zależnie od projektu oraz wymagań builda liczna slave'ow może się różnić. Slavy mogą być uruchomione na różnych systemach operacyjnych i zależnie od wymagań builda, master wybiera odpowiedniego slavea do wykonania builda i testów. 
Do głównych zadań slave'a należą:
\begin{itemize}
    \item nasłuchiwanie na polecenia Mastera,
    \item wykonie jobów zleconych przez Mastera,
    \item developerzy mogą "ręcznie" wybrać slave na którym ma zostać wykonane zadanie ale z reguły Master dobiera najbardziej pasujący slave.
\end{itemize}

\begin{figure}[htbp]
    \centering
    \includegraphics[width=10cm]{master-slave.png}
    \caption{master-slave architektura}
    \label{fig:master-slave}
\end{figure}

Jak do tej pory pokrótce opisaliśmy za co odpowiedzialne są poszczególne komponenty w Jenkinsie. Przedstawimy teraz przykładową architekturę oraz opiszemy za co są odpowiedzialne poszczególne jej elementy 

\begin{figure}[htbp]
    \centering
    \includegraphics[width=10cm]{archritektura-przyklad.png}
    \caption{architektura przykład}
    \label{fig:jenkins-architektura}
\end{figure}

\begin{itemize}
    \item Developer zapisuje zmiany w kodzie na zewnętrznym repozytorium,
    \item master jest połączony z repozytoriom i regularnie sprawdza czy pojawiły się jakieś zmiany. Wszystkie slave'y są połączone z masterem,
    \item master otrzymuje żądanie wykonania zadania, które zostaje przekazane do odpowiedniego slave'a,
    \item slave wykonuje zlecone zadania, generuje raporty testów. Master ciągle monitoruje wyniki testów.
\end{itemize}

W dalszej części pracy zaprezentujemy przykładowe zastosowanie tego narzędzia. 

\subsection{Aplikacja}

Aplikacja implementuje REST api i będzie wyświetlała imię użytkownika podane do path URL. Projekt posiada proste pliki java, w których jest umieszczona logika naszej aplikacji oraz pliki Maven'a do budowania naszej aplikacji. Tak wygląda kod pliku pom.xml:

\begin{lstlisting}
    <?xml version="1.0" encoding="UTF-8"?>
<project xmlns="http://maven.apache.org/POM/4.0.0" xmlns:xsi="http://www.w3.org/2001/XMLSchema-instance"
	xsi:schemaLocation="http://maven.apache.org/POM/4.0.0 https://maven.apache.org/xsd/maven-4.0.0.xsd">
	<modelVersion>4.0.0</modelVersion>
	<parent>
		<groupId>org.springframework.boot</groupId>
		<artifactId>spring-boot-starter-parent</artifactId>
		<version>2.3.2.RELEASE</version>
		<relativePath/>
	</parent>
	<groupId>com.example</groupId>
	<artifactId>rest-service</artifactId>
	<version>0.0.1-SNAPSHOT</version>
	<name>rest-service</name>
	<description>Demo project for Spring Boot</description>

	<properties>
		<java.version>1.8</java.version>
	</properties>

	<dependencies>
		<dependency>
			<groupId>org.springframework.boot</groupId>
			<artifactId>spring-boot-starter-web</artifactId>
		</dependency>

		<dependency>
			<groupId>org.springframework.boot</groupId>
			<artifactId>spring-boot-starter-test</artifactId>
			<scope>test</scope>
			<exclusions>
				<exclusion>
					<groupId>org.junit.vintage</groupId>
					<artifactId>junit-vintage-engine</artifactId>
				</exclusion>
			</exclusions>
		</dependency>
	</dependencies>

	<build>
		<plugins>
			<plugin>
				<groupId>org.springframework.boot</groupId>
				<artifactId>spring-boot-maven-plugin</artifactId>
			</plugin>
		</plugins>
	</build>

</project>

\end{lstlisting}

W pliku tym znajdują się zależności jak i wtyczki Spring Boot potrzebne do działania naszej aplikacji. 

\subsection{Dockerfile} 

Dockerfile jest to specyficzny plik, który pozwala nam zdefiniować jak powinien wyglądać nasz kontener. Każda linia w Dockerfile to osobna instrukcja, która opisuje jak powinien wyglądać końcowy kontener. 

Na początku konieczne jest zbudowanie naszego programu, żeby można było go wykorzystać w naszym dockerfile. W tym celu użyjemy komendy ./mvnw clean package, która skompiluje nasz kod i spakuje go do pliku wykonywalnego rest-service-0.0.1-SNAPSHOT.jar. Plik ten będzie wykorzystywany w naszym Dockerfile. 

\begin{lstlisting}
    FROM openjdk:8-jdk-alpine
    VOLUME /tmp
    ADD target/rest-service-0.0.1-SNAPSHOT.jar app.jar
    ENTRYPOINT ["java","-jar","app.jar"]
    EXPOSE 2222
\end{lstlisting}


Każda linia tego pliku dodaje dodatkową funkcjonalność do naszego projektu, więc warto wyjaśnić co w każdej linii się znajduje. W pierwszej linii importujemy dostępny w oficjalnym repozytorium Dockera linuxowy obraz alpine wraz z zainstalowanym na nim openjdk. Alpine Linux jest to podstawowy system operacyjny charakteryzujący się prostotą oraz małym rozmiarem pojemności dyskowej jaką zajmuje. Nie posiada on zbędnych bibliotek, które niepotrzebnie zajmowałyby miejsce na naszym kontenerze, stąd też nasz wybór padł właśnie na ten kontener. Następnie dodajemy wcześniej spakowany plik jar, który znajduje się w folderze /target. Kolejno zaznaczamy jaka komenda powinna zostać uruchomiona po uruchomieniu kontenera oraz udostępniamy port 2222 do dostępu publicznego. 

Kolejno w konsoli użyliśmy trzech komend, aby zbudować nasz projekt i uruchomić go w kontenerze. 
\begin{lstlisting}
    mvn clean install 
    docker build -t pracainzynierka
    docker run pracainzynierska -p 2222:2222
\end{lstlisting}
W pierwszej kolejności lokalnie budujemy projekt by zaktualizować nasz plik jar, który jest nam potrzebny podczas budowania obrazu Dockera w drugiej komendzie. W ostatnim kroku uruchamiamy nasz kontener. Po tych krokach wchodząc pod adres http://127.0.0.1:2222/greeting otrzymamy powitalną odpowiedź z naszego kontenera. 
W dalszej części pracy inżynierskiej zautomatyzujemy ten proces przy pomocy Jenkinsa. 

\subsection{Automatyzacja przy użyciu Jenkinsa}

Jako, że użyliśmy dockera by uruchomić naszą aplikację lokalnie, nic nie stoi na przeszkodzie by również użyć Dockera do pracy z Jenkinsem. Problemem jaki napotkaliśmy podczas implementacji tego rozwiązania polegał na braku komend dockera wewnątrz kontenera, dlatego trzeba było dodać kilka warstw do naszego Jenkinsowego Dockerfile by umożliwić taką funkcjonalność. 

Finalna wersja pliku Dockera wygląda następująco: 

\begin{lstlisting}
    from jenkins/jenkins:lts
    USER root
    RUN apt-get update -qq \
    && apt-get install -qqy apt-transport-https ca-certificates curl gnupg2 software-properties-common
    RUN curl -fsSL https://download.docker.com/linux/debian/gpg | apt-key add -
    RUN add-apt-repository \
   "deb [arch=amd64] https://download.docker.com/linux/debian \
   $(lsb_release -cs) \
   stable"
    RUN apt-get update  -qq \
    && apt-get install docker-ce=17.12.1~ce-0~debian -y
    RUN usermod -aG docker jenkins
\end{lstlisting}

Kolejno przy użyciu dwóch kolejnych komend:
\begin{lstlisting}
    docker image build -t jenkins-docker .
    docker container run -d -p 8080:8080 -v /var/run/docker.sock:/var/run/docker.sock jenkins-docker
\end{lstlisting}

jesteśmy w stanie wchodząc pod adres 127.0.0.1:8080 finalnie dostać się do Jenkinsa

\subsection{Konfigurowanie Jenkinsa}

Głównymi narzędziami, które będzie trzeba skonfigurować jest JDK, Maven oraz GIT, by móc budować aplikacje oraz klonować kod z repozytorium. Wszystkie kroki wykonuje się z poziomu Jenkinsa. W naszym projekcie konfiguracja wygląda jak na zrzutach ekranu poniżej:

\begin{figure}[htbp]
    \centering
    \includegraphics[width=10cm]{iz-JKD-Git.png}
    \caption{JDK-Git}
    \label{fig:JDK-Git}
\end{figure}
\begin{figure}[htbp]
    \centering
    \includegraphics[width=10cm]{iz-maven.png}
    \caption{maven}
    \label{fig:maven}
\end{figure}
\begin{figure}[htbp]
    \centering
    \includegraphics[width=10cm]{iz-Jenkins-port.png}
    \caption{Jenkins port}
    \label{fig:Jenkins-port}
\end{figure}
\begin{figure}[htbp]
    \centering
    \includegraphics[width=10cm]{iz-pipline.png}
    \caption{pipline}
    \label{fig:pipeline}
\end{figure}


\subsection{Jenkinsfile} 
Jenkinsfile jest plikiem, w którym definiujemy wszystkie kroki, które mają zostać podjęte w ramach Jenkins pipeline. Mamy również możliwość wyboru, które zadania na jakich slave'ach mają zostać wykonane oraz inne metody konfiguracji pipline'u.

W naszym projekcie plik ten składa się z czterech następujących kroków:

\begin{lstlisting}
    node {  
        
	    def mvnHome = tool 'maven-3.6.3'
	    def dockerImage
	    def dockerImageTag = "pracainzynierka${env.BUILD_NUMBER}"
		def DOCKER_FILES_DIR = "./initial"
		def dockerfile = "Dockerfile"
	    
	    stage('Clone Repo') { // for display purposes
	      git 'https://github.com/patrmus054/papryk-inzynier.git'           
	      mvnHome = tool 'maven-3.6.3'
	    }    
	  
	    stage('Build Project') {
	      sh "'${mvnHome}/bin/mvn' clean install -f ./initial/pom.xml"
	    }
			
	    stage('Build Docker Image') {
	      dockerImage = docker.build("pracainzynierka:${env.BUILD_NUMBER}", "-f ${DOCKER_FILES_DIR}/${dockerfile} ${DOCKER_FILES_DIR}")
	    }
	   
	    stage('Deploy Docker Image'){
	      echo "Docker Image Tag Name: ${dockerImageTag}"
		  sh "docker run pracainzynierka:${env.BUILD_NUMBER} -p 2222:2222 "
	    }
}
\end{lstlisting}

Na początku pliku definiujemy zmienne lokalne, potem w kolejnych krokach zasadniczo wszystkie kroki, które musieliśmy wcześniej wpisywać "ręcznie": pobranie projektu z repozytorium, budowanie aplikacji, budowanie obrazu Dockera i uruchamianie aplikacji. 

\subsection{Podsumowanie projektu}

W dzisiejszym środowisku IT mamy mnogość narzędzi które pozwalają w stosunkowo prosty sposób automatyzować procesy związane z inżynierią oprogramowania i nie tylko. Jenkins jest w branży od pewnego czasu i dzięki rozbudowanemu ekosystemowi mamy możliwość automatyzować rzeczy, które wcześniej zajmowały dużo czasu. W projekcie wykorzystaliśmy możliwości narzędzia do stworzenia dwóch działających kontenerów. Rozwiązanie ma jednak jedynie charakter prezentacyjny i nie powinno stanowić inspiracji do produkcji przemysłowej. 

\section{Platformy SaaS z wbudowanym CI/CD}
tutaj mógłbym rozwinąć rozdział 3 i powiedzieć, że coraz mniej używa się Jenkinsa i obecnie popularne są platformy jak Github Actions, CircleCI czy gitlabCI, które są trigerrowane przez commity, pushe czy też tworzenie tagów z pomocą systemu wersji git. Myślę, że mógłbym tutaj zrobić nawiązanie do części praktycznej pracy:
4a) continues delivery w React Native  na przykładzie GitHub actions
4b) continues deploymeny na przykładzie aplikacji backendowej w javaScriptcie
4c) continues deployment strony internetowej hostowanej przez GitHub actions za pomocą circleCI
W każdym z tych podrozdziałów opisałbym jaki jest ogólny problem i dalej opisałbym już konkretne rozwiązanie
\section{Testy a continuous integration}
W celu lepszego zrozumienia istoty automatyzacji testów konieczne jest najpiew zrozumienie testów samych w sobie. Jak zostało to opisane w rozdziale 1.4 - pisanie testów jest integralną częścią pracy każdego programisty. Wiele osób uważało to dawniej za żmudne zadanie, nieprzynoszące wymiernych korzyści, jednakże z biegiem czasu stało się jasne, że w dużych projektach informatycznych są one konieczne, co widać w dzisiejszych czasach w wypowiedziach wielu osób. \cite{UnitOpinions} \cite{UnitResults}
\begin{figure}[htbp]
    \centering
    \includegraphics[width=10cm]{images/tdd.png}
    \caption{Zasada Red-Green-Refactor}
    \label{fig:redgreen}
\end{figure}
\par Najpopularniejsze obecnie są dwie metody pisania testów: 
\begin{itemize}
    \item TDD - Test Driven Development - zaproponowana została przez Kenta Becka w 2002 roku. \cite{TestDrivenDevelopment} Zakłada, że testy będą napędzały tworzenie projektu i stały u jego podstawy. 
    \par Pierwszym krokiem przy implementacji nowej funkcji w naszym projekcie powinno być napisanie samego testu, który oczywiście na początku nie powiedzie się, ponieważ kod, który ma on testować nie został jeszcze napisany. Po napisaniu testu przechodzi się do pisania kodu, który w jak najłatwiejszy sposób będzie w stanie zaliczyć napisany przez nas test. Ostatnim krokiem tego cyklu jest refactoring kodu, polepszający jego jakość, tak aby spałniał on oczekiwane standardy.
    Takie podejście pozwala na tworzenie dobrze zaprojektowanego kodu, który jest w całości pokryty testami, co owocuje w przyszłości, kiedy konieczne jest wprowadzanie zmian w bazie kodu.
    
    \item BDD - Behavior Driven Development - metodologia zaproponowana przez Dana Northa w 2006 roku. Zakłada ona zaangażowanie w tworzenie oprogramowania osób nietechnicznych - analityków oraz klientów dla których oprogramowanie jest tworzone. Testy tworzone są według zasady "Given - When - Then (Zakładając - Gdy - Wtedy)". Można w ten sposób łatwo opisać testy, które każdy będzie w stanie zrozumieć, a następnie zaimplementować przy użyciu odpowiedniego frameworka, np. JBehave dla Java lub Behave dla Pythona. 
\end{itemize}

\begin{figure}[htbp]
    \centering
    \includegraphics[width=10cm]{images/testing_triangle.png}
    \caption{Piramida testów}
    \label{fig:testing}
\end{figure}



\subsection{Testy jednostkowe}
Testem jednostkowym nazywa się kod, który jest w stanie wywołać inny fragment kodu programu, a nastepnie sprawdzić czy działanie tamtego kodu jest zgodne z zakładanym przez programistę działaniem. \cite{UnitDefinition}
\par Autor tej definicji zdefiniował też kilka warunków, które powinien spełniać każdy test jednostkowy: 
\begin{itemize}
    \item jest w pełni zautomatyzowany
    \item może być wykorzystywany w wielu miejscach, także jako część innych testów
    \item działa w pamięci, bez dostępu do baz danych lub plików
    \item jest deterministyczny, nie zawiera losowych danych
    \item jest szybki
    \item skupia się na pojedynczym logicznym elemencie programu
    \item czytelny
    \item łatwy w zrozumieniu
    \item wiarygodny 
\end{itemize}
Fragmentem kodu, który podlega testom jest zwykle najmniejszy jego fragment, który odpowiada za jedno logiczne działanie. Najczęściej będzie to pojedyczna metoda klasy, cała klasa lub rzadziej nawet kilka klas. Odpowiednie przygotowanie testów jest kluczowe jeśli chcemy uniknąć w naszym projekcie regresji - pojawienia się błędów w kodzie, który wcześniej działał poprawnie. 

\begin{lstlisting}[caption={Test jednostkowy w języku Python}]
    
    TODO przykladowy kod
\end{lstlisting}
\subsection{Testy integracyjne}

\subsection{Testy end-to-end}

\subsection{Rola CI w testach}

\subsection{Przykład w GitHub Actions}
\section{Podsumowanie}
myślę, że cały wydźwięk podsumowania powinien być nastawiony na to że dzięki CI/CD programiści mogą efektywnie działać, ich kod szybko znajduje się na wersji produkcyjnej i dzięki odpowiednio skonfigurowanemu workflow mogą być bardziej pewni, że ich zmiany nie zepsują istniejących funkcjonalności


\begin{thebibliography}{12}

\bibitem{AgileManifesto} Kent Beck; James Grenning; Robert C. Martin; Mike Beedle; Jim Highsmith; Steve Mellor; Arie van Bennekum; Andrew Hunt; Ken Schwaber; Alistair Cockburn; Ron Jeffries; Jeff Sutherland; Ward Cunningham; Jon Kern; Dave Thomas; Martin Fowler; Brian Marick (2001). "Principles behind the Agile Manifesto"
\bibitem{DevOpsBook} Bass, Len; Weber, Ingo; Zhu, Liming (2015). DevOps: A Software Architect's Perspective

\end{thebibliography}
\end{document}
